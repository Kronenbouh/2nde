\input{preamble.tex}

\begin{document}
\pagestyle{fancy}
\fancyhead[L]{Seconde}
\fancyhead[C]{\textbf{DS n°1 — Ensembles de nombres}}
\fancyhead[R]{\today}

\reversemarginpar

\null\vspace{-30pt}
Nom / Prénom : \\

Consignes particulières : 
\begin{itemize}[label=$\bullet$]
	\item 
	La calculatrice est {interdite}.
	\item 
	L'exercice \ref{exe:1} peut être fait entièrement sur la feuille d'évaluation. Écrire son nom avant de rendre le sujet pour qu'il soit corrigé.
	\item 
	Toute trace de recherche est prise en compte.
	\item 
	L'abbréviation $\tq$ signifie ``tel que''.
\end{itemize}

\hrule

\exe{3}{
	Vrai ou faux ? Cocher la case correspondante (aucune justification n'est attendue). \\
	\vspace{-20pt}
	\begin{center}
	\begin{tabular}{c c c}
		\hspace{10cm} & Vrai & Faux \\
		$\Z \subset \Q$  & $\square$ & $\square$  \\
		$\dfrac5{14} + \dfrac17$ est un nombre décimal & $\square$ & $\square$  \\
		$\dfrac19 = 0,11111$ & $\square$ & $\square$  \\ \vspace{-5pt} \\
		L'encadrement $0,75 < 0,758 < 0,76$ est à $10^{-2}$ près & $\square$ & $\square$  \\
		$\sqrt{144}$ est un nombre rationnel & $\square$ & $\square$  \\
		$ \{-4, 0, 1, 7 \} \subset \{ n \in \N \tq -4 < n \leq 8 \}$ & $\square$ & $\square$  \\
	\end{tabular}
	\end{center}
	}{exe:1}{
	
	Les réponses sont, dans l'ordre : vrai, vrai, faux, vrai, vrai, faux.
	}
	
\exe{3}{
	\begin{enumerate}
		\item Sans justification, donner un exemple de nombre rationnel \textbf{qui n'est pas décimal}.
		\item Sans justification, donner un exemple de nombre réel \textbf{qui n'est pas rationnel}.
		\item Sans justification, donner un exemple de nombre décimal tel que, 
		multiplié par une certaine puissance de 10, ce nombre est pair.
	\end{enumerate}
}{exe:2}{
Les nombres suivants convienent :
	\begin{enumerate}
		\item $\dfrac13$ ;
		\item $\sqrt{2}$ ;
		\item 2.
	\end{enumerate}
}

\exe{3}{
Mettre les nombres ci-dessous sous forme de fractions :
\begin{multicols}{3}
\begin{enumerate}[label=(\alph*)]
\item 0,0025
\item 0,111000111
\item 0,120051200512005... (se répète à l'infini)
\end{enumerate}
\end{multicols}
}{exe:3}{
Attention, les deux premiers nombres sont simplement décimaux.
\begin{enumerate}[label=(\alph*)]
\item $0,0025 = \dfrac{25}{1000}=\dfrac{1}{40}$
\item $0,111000111 = \dfrac{111000111}{10^9}$
\item On note $x=0,120051200512005...$, on a $x \times 10^5 - x = 12005$ donc $x=\dfrac{12005}{99999}$.
\end{enumerate}
}

\exe{2}{
\begin{enumerate}
\item Donner un élément appartenant à chacun des ensembles ci-dessous :
\begin{multicols}{2}
\begin{enumerate}[label=(\alph*)]
\item $A = \{ n \in \Z \tq -7 \leq n < 8 \}$.
\item $B = \{ n \in \N \tq n \leq 7  \}$. 
\end{enumerate}
\end{multicols}
\item A-t-on $A \subset B$ ? $B \subset A$ ? \textbf{Justifier}.
\end{enumerate}
}{exe:4}{
\begin{enumerate}
\item Donner un élément appartenant à chacun des ensembles ci-dessous :
\begin{multicols}{2}
\begin{enumerate}[label=(\alph*)]
\item $2 \in A$.
\item $0 \in B$. 
\end{enumerate}
\end{multicols}
\item On a $B= \{0,1,2,3,4,5,6,7 \}$ donc clairement $B \subset A$. En revanche $-7 \not \in B$ et $-7 \in A$ donc $A \not \subset B$. 
\end{enumerate}
}

\exe{3}{
Donner le développement décimal de $\frac1{11}$. Ce développement décimal est-il périodique ? Si oui, de quelle longueur est la période ? Justifier.
}{exe:5}{
$\frac1{11}=0,09090909...$, ce développement décimal est périodique, la période (09) est de longueur 2.
}

\exe{4}{
	Encadrer les nombres suivants (par des nombres décimaux) à l'amplitude demandée.
	\begin{multicols}{2}
	\begin{enumerate}[label=\alph*)]
		\item 4,75 à $10^{-1}$ près
		\item $1024,75 \times 10^{-2}$ à $10^{-2}$ près.
		\item $1,0001~0001 \dots$ à $10^{-5}$ près.
		\item -2,95678 à $10^{-3}$ près.
	\end{enumerate}
	\end{multicols}
}{exe:exo6}{
	\begin{enumerate}[label=\alph*)]
		\item $4,7 < 4,75 < 4,8$.
		\item $10,24 < 10,2475 < 10,25$.
		\item $1,0001 < 1,0001~0001 \dots < 1,0001~1$.
		\item $-2,957 < -2,95678 < -2,956$.
	\end{enumerate}
}

\exe{3}{
Démontrer que $\dfrac{1}{30} = \dfrac{1}{10} \cdot \dfrac13$. En déduire que $\dfrac{1}{30}$ n'est pas un nombre décimal. \\
\textit{Indication : on pourra utiliser le résultat du cours sur le nombre $\dfrac13$.}
}{exe:exo7}{
Supposons que $\dfrac{1}{30}$ soit décimal, on écrit alors $\dfrac{1}{30} = \dfrac{a}{10^n}$ avec $a \in \Z$ et $n \in \N$. 
Par suite, $\dfrac13 = \dfrac{a}{10^{n-1}}$, donc $\dfrac13$ est décimal, ce qui est faux (voir démonstration du cours).
}

\exe{3}{
\begin{enumerate}
\item
Rappeler la définition du nombre $\dfrac{a}{b}$ pour $a, b \in \Z$ et $b$ non nul.
\item
En utilisant \textbf{leurs définitions} démontrer que les nombres $\dfrac34$ et $\dfrac68$ sont égaux.
\end{enumerate}
}{exe:exo8}{
\begin{enumerate}
\item
$\dfrac{a}{b}$ est \textbf{l'unique} nombre tel que $\dfrac{a}{b} \times b =a$.
\item
$\dfrac{3}{4}$ est \textbf{l'unique} nombre tel que $\dfrac{3}{4} \times 4 =3$.
$\dfrac{6}{8}$ est \textbf{l'unique} nombre tel que $\dfrac{6}{8} \times 8 =6$.
On en déduit que $\dfrac{3}{4} \times 4 \times 2 = \dfrac34 \times 8 = 6$, par unicité, on conclut que $\dfrac34 = \dfrac68$.
\end{enumerate}
}

\exe{3}{
Déterminer le plus petit ensemble de nombres auquel chaque nombre ci-dessous appartient (par exemple, $1 \in \N$).
\begin{multicols}{3}
\begin{enumerate}[label=(\alph*)]
\item $\dfrac{81}{9}$
\item $\sqrt{2}$
\item $0,123~123~\dots$
\item $\dfrac16 + \dfrac13$
\item $10^2-700$
\item $\dfrac19 \times (-9) + 5$  
\end{enumerate}
\end{multicols}
}{exe:exo9}{
\begin{multicols}{3}
\begin{enumerate}[label=(\alph*)]
\item $\dfrac{81}{9}=9 \in \N$
\item $\sqrt{2} \in \R$
\item $0,123~123~\dots \in \Q$
\item $\dfrac16 + \dfrac13 = \dfrac12 \in \D$
\item $10^2-700 = -600 \in \Z$
\item $\dfrac19 \times (-9) + 5 = 4 \in \N$  
\end{enumerate}
\end{multicols}
}

\exe{}{
(Exercice bonus) \\
Construire deux ensembles $A$ et $B$ contenant chacun une infinité de nombres réels et tels que $A \not \subset B$ et $B \not \subset A$.
}{exe:exo10}{
$A = \{ 2n \tq n \in \N \}$ et $B=\{2n+1 \tq n \in \N\}$. $A$ est l'ensemble des entiers pairs et $B$ l'ensemble des entiers impairs. On a bien $A \not \subset B$ et $B \not \subset A$.
}

%%%%%%%%%%%%

\newpage
\fancyhead[C]{\textbf{Solutions}}
\shipoutAnswer

\end{document}
