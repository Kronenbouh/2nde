%Ensembles de nombres
\newcommand{\N} {\mathbb{N}}
\newcommand{\Ne}{\N^\ast}
\newcommand{\Z} {\mathbb{Z}}
\newcommand{\D} {\mathbb{D}}
\newcommand{\Q} {\mathbb{Q}}
\newcommand{\R} {\mathbb{R}}
\newcommand{\Rb}{\overline{\mathbb{R}}}
\newcommand{\Rp}{\R_+}
\newcommand{\Rm}{\R_-}
\newcommand{\K} {\mathbb{K}}
\newcommand{\Cx}{\mathbb{C}}

%Opérateurs
\newcommand{\equi}{\Leftrightarrow}

%Normes
\DeclarePairedDelimiter\abs{\lvert}{\rvert}

%Flemme
\newcommand{\tq}{\text{ t.q. }}
\newcommand{\bigset}[1]{\bigl\{ #1 \bigr\}}
\newcommand{\Bigset}[1]{\Bigl\{ #1 \Bigr\}}

%Commande d'exo
\newcommand{\exe}[4]{
	\begin{Exercise}[title=#1, label=#3]
		#2
	\end{Exercise}
	\begin{Answer}[ref=#3]
		#4
	\end{Answer}
}

\newcommand{\exemulticols}[5]{
	\begin{multicols}{2}
	\begin{Exercise}[title=#1, label=#4]
		\if\relax\detokenize\expandafter{\ExerciseTitle}\relax
		%\marginnote{[Bonus]}
		\else
		\marginnote{\mbox{[\ExerciseTitle]\qquad}}
		\addtocounter{points}{\ExerciseTitle}
		\fi
		#2
	\end{Exercise}
	\columnbreak
		#3
	\end{multicols}
	\begin{Answer}[ref=#4]
		#5
	\end{Answer}
}