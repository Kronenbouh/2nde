\input{preamble}
\newcommand{\C}{\mathcal{C}}

\begin{document}

\pagestyle{fancy}
\fancyhead[L]{Seconde}
\fancyhead[C]{\textbf{TD n°4 : fonctions}}
\fancyhead[R]{\today}


\exe{}{
On considère le programme de calcul suivant :
\begin{enumerate}[(label=\roman*)]
\item On prend un nombre entier (positif ou négatif).
\item On divise ce nombre par 3.
\item On ajoute 1 à ce nombre.
\end{enumerate}
Peut-on modéliser ce programme de calcul par une fonction ? Si oui, donner :
\begin{enumerate}
\item Son domaine de définition
\item Sa forme algébrique.
\item L'image du point $1$ par la fonction.
\item Les éventuels antécédent du point $0$ par la fonction.
\end{enumerate}
}{exe:1}{

}

\exe{, difficulty=0}{
	Un étudiant jette une balle dans les airs et mesure la hauteur de la balle tous les quarts de seconde.
	Il note ses résultats dans le tableau ci-dessous.
	\begin{center}
		\def\arraystretch{1.5}
		%\setlength\tabcolsep{20pt}
		\begin{tabular}{|c|c|c|c|c|c|c|c|}\hline
			Hauteur (cm) & 85 & 145 & 190 & 145 & 85 & 40 & 0 \\ \hline
			Temps (s) & 0 & 0,25 & 0,5 & 0,75 & 1 & 1,25 & 1,5 \\\hline
		\end{tabular}
	\end{center}
	
	\begin{enumerate}
		\item La hauteur est-elle une fonction du temps ? Justifier.
		\item Le temps est-il une fonction de la hauteur ? Justifier.
	\end{enumerate}
}{exe:2}{
	\begin{enumerate}
		\item On peut associer \textbf{une seule} hauteur à chaque temps. La hauteur peut donc être vue comme fonction du temps.
		\item On ne peut pas associer un unique temps à chaque hauteur. Par exemple, il y a deux temps distincts pour lesquels la hauteur est de $85$cm (0s et 1s).
		Le temps ne peut donc pas être vu comme fonction de la hauteur.
	\end{enumerate}
}

\exe{}{
On considère la fonction $f$ définie par :
\[ \begin{array}{c c c c}
f : & \R & \to & \R \\
& x & \mapsto & 3x -2
\end{array} \]
\begin{enumerate}
\item Tracer la courbe représentative de $f$ dans un repère orthonormé.
\item Donner le domaine de définition de la fonction $f$. La courbe tracée en 1/ couvre-t-elle tout ce domaine ? 
\item Donner l'image du point $1$ par la fonction $f$.
\item Donner Les éventuels antécédent du point $0$ par la fonction $f$.
\end{enumerate}
}{exe:3}{

}

\exe{}{
$x$ est une variable réelle.
\begin{enumerate}
\item Traduire l'appartenance de $x$ à chaque intervalle ci-dessous en termes d'inégalités :
\begin{multicols}{2}
\begin{enumerate}
\item $I_1 = ]-3;7[$
\item $I_2 = [4;12]$
\item $I_3 = ]-\infty;0]$
\item $I_4 = [-3;4[$
\end{enumerate}
\end{multicols}
\item En utilisant le résultat de la question précédente, donner une condition pour que $x$ appartienne à l'intervalle $I_1$ \textbf{ET} à l'intervalle $I_4$.
Exprimer ce résultat sous forme d'intervalle.
\end{enumerate}
\textit{Pour deux ensembles quelconques, $A$ et $B$, on note $A \cap B$ leur intersection. \newline
On a $(x \in A$ ET $x \in B) \equi (x \in A \cap B)$}.
}{exe:4}{

}

\exe{, difficulty=1}{

On considère les fonctions $f$ et $g$ définies par :
\[ \begin{array}{c c c c c c c c c}
f : & \R & \to & \R & \qquad & g : & \R & \to & \R \\
& x & \mapsto & 2x+3 & \qquad & & x & \mapsto & 1-x
\end{array} \]

\begin{enumerate}
\item Tracer une partie des courbes $\C_f$ et $\C_g$ dans un repère orthonormé.
\item Déterminer les images du point $x=0$ par les fonctions $f$ et $g$. Que remarquez-vous ?
\item Déterminer les éventuels antécédents du point $y=0$ par les fonctions $f$ et $g$.
\item Existe-t-il un point $x$ tel que $f(x)=g(x)$. Si oui, le déterminer. Si non, expliquer.
\end{enumerate}

}{exe:5}{

}

\exe{,difficulty=1}{
	Un fonction $f$ admet le tableau de valeurs suivant.
		\begin{center}
		\def\arraystretch{1.2}
		\setlength\tabcolsep{20pt}
		\begin{tabular}{|c|c|c|c|c|}\hline
			$x$ & 0 & -2 & 1 & -1 \\ \hline
			$f(x)$ & 1 & 0 & 0 & 1 \\ \hline
		\end{tabular}
		\end{center}
	Parmis les expressions algébriques suivantes, lesquelles \textbf{ne peuvent pas} correspondre à $f(x)$ ?
		\begin{multicols}{4}
		\begin{enumerate}[label=\roman*)]
			\item $1-x$
			\item $1+\dfrac{x}2$
			\item $\dfrac{1-x}2$
			\item $\dfrac{-x^2 - x + 2}2$
		\end{enumerate}
		\end{multicols}
}{exe:6}{

}

\exe{, difficulty=1}{
	Déterminer une fonction numérique $g$, définie sur $\R$ et telle que :
	\[ g(0)=5 \qquad g(3)=0 \]
	Peut-on trouver d'autres fonctions définies sur $\R$ et vérifiant ces deux conditions ? 
}{exe:7}{

}

\exe{, difficulty=2}{
On rappelle la formule de double distributivité, soient $a, b, c$ et $d$ quatre nombres réels quelconques. On a :
\[ (a+b)(c+d) = ac + ad + bc + bd . \]
En particulier, si $a=c$ et $b=d$, on obtient l'identité remarquable :
\[ (a+b)^2 = a^2 + 2ab + b^2. \]
Ces formules sont à connaître \textbf{par coeur}. \newline
\begin{enumerate}
\item Donner une expression développée de $(a-b)^2$. \newline
\textit{Indication : remplacer $b$ par $-b$}.
\item On considère la fonction :
\[ \begin{array}{c c c c}
f : & \R & \mapsto & \R \\
& x & \to & x^2 - 4x +10
\end{array} \]
Démontrer que $f(x)=6+(x-2)^2$.
\item En conclure que, pour tout $x \in \R, f(x) \geq 6$.
\item Donner un nombre $x_0 \in \R$ tel que $f(x_0)=6$.
\end{enumerate}
}{exe:8}{

}

%\exe{, difficulty=2}{
%
%}{exe:9}{
%
%}




\end{document}
